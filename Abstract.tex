%% LyX 2.0.5.1 created this file.  For more info, see http://www.lyx.org/.
%% Do not edit unless you really know what you are doing.

\chapter*{Abstract}

In this paper we investigate the issue of resource matching between
jobs and machines in Intel's compute farm.
We show that common heuristics such as \bef\ and \wof\ may fail to
properly utilize the available resources when applied to either cores
or memory in isolation.
In an attempt to overcome the problem we propose \mif, a heuristic
which attempts to balance usage between resources.
While this indeed usually improves upon the single-resource heuristics,
it too fails to be optimal in all cases.
As a solution we default to \maj, a meta-heuristic that employs all
the other heuristics as sub-routines, and selects the one which
matches the highest number of jobs.
Extensive simulations that are based on real workload traces from
four different Intel sites demonstrate that \maj\ is indeed the most
robust heuristic for diverse workloads and system configurations, and
provides up to 22\% reduction in the average wait time of jobs.

